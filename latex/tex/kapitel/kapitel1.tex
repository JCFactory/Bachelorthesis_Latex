\chapter{Use of RFID Technology}
\label{Kap1}

The following chapter will discuss the reasons why choosing the technology and the scope of the RFID technology. It will start by explaining fundamental information and functionalities of RFID readers, tags and further equipment to build a RFID application. After that, some 'State of the Art' applications and use cases will be shown. In the end, there will be mentioned some large companies which provide medical RFID solutions.

\section{Motivation}

Concerning the organization and management of medical devices or patients in a hospital, there exist many problems. In the following, some examples of these problems as described by Ajami and Rajabzadeh \cite{ncbi} will be given.

\subsection{Decision Making}

First of all, when it comes to decision making, e.g. about the correct treatment of a severe illness, many physicians are stumped for an answer or their opinions are divided. To enable a rapid diagnosis and to improve the patient's health status, 'smart healthcare' \cite{henrici} would help a lot. For instance, RFID tags which are equipped with sensors to ensure the effectiveness of medicine accelerate the treatment process a lot. Furthermore, patients or hospital beds equipped with RFID tags make it easier to identify and manage the amount of patients as well as the workflow.

\subsection{Internal Communication}

Secondly, poor communication between nurses and physicians deteriorates medical supply. For instance, if a nurse notices that a patient needs a more tranquilizer because he became very nervous, she has to tell the doctor to dose the patient with the correct amount of tranquilizer. But often, a physician is occupying another patient. So, the communication is one problem but the other problem is the staff shortage. Thus, inadequate patient monitoring is emerges. And sometimes, there is the risk of misidentification of patients. To explain the last point, there is an easy example: At the urology department are two elder patients, Paul Schmitt and Jochen Schmitt. They are not brothers or related to each other and suffer from different types of illness. Paul suffers from kidney insufficiency whereas Jochen suffers from prostatic lithiasis. The first one needs a dialysis every day whereas the second one needs a radiosurgery. Because both patients are unable to walk themselves, nurses and clinical staff have to bring them to the particular treatment room. The problem should be easy to understand, both patients have the same surname but need completely different treatments. If the treatments would be commuted, their health status would deteriorate and they might die because of the misidentification.

\subsection{Investment possibilities}

Another important point for hospitals is the budget and their possibilities to investigate in new technologies which makes the enrollment of a new RFID system more challenging. Furthermore, clinical staff and phycisians have to be introduced into the new technologies. Not only the human factor plays a significant role but also the existing systems, such as the \ac{HIS}, \ac{RIS} or \ac{LIS}. If a new identifying system or software should be integrated into a hospital or healthcare institution, it has to be deployed suitably to the existing system architecture. To achieve the last point, Ajami and Rajabzadeh \cite{ncbi} recommend starting with small RFID projects and mentions countermeasures to increase the acceptance of such applications by healthcare institutions. To give an example, the regulations to protect patient's privacy should be mature to achieve more institutional support. Besides, there should exist more customized RFID systems which accomplish the individual tasks of their users.  

\subsection{Medication Administration System}

To explain the positive impact of using RFID systems, in the following, a few applications will be described briefly (see also \cite{ncbi}). Firstly, Ajami and Rajabzadeh describe a Medication Administration System which automatically verifies medication and generates the corresponding medication medicine. There exists multiple intents of developing a Administration System, such as preventing human errors (like for example mislabeling of tissue specimens in gastrointestinal and colorectal surgery endoscopy units). The second most common error which occured were that patients have been labelled incorrectly. To avoid these errors, an initiative of developing an application of RFID technology to specimen bottles was started. The aim of this initiative was to create a paperless pathology requisition system and the correct confirmation by both the endoscopy nursing staff as well as the endoscopist for each specimen bottle. As a result of deploying the application, specimen-labeling errors were significantly reduced.

\subsection{Wisely Aware RFID Dosage}

Another RFID system, called \ac{WARD} system should prevent the risk of medication errors triggered by medical staff. It is based on integrated barcode and RFID tags which should demonstrate effective and safe patient care environment. 
To give another example of RFID applications, the following paragraph will describe the \ac{MIMS} which includes a mobile nursing care system using RFID technology. There are many implemented functionalities in the MIMS, such as the tracking of patient's vital signs across various locations and in different medical facilities. The vital sign monitoring enables medical staff to watch critical ill patients carefully and permanently and reduces the risk of serious harm resulting from slow provision \cite{ncbi}. The mobile application MIMS offers alarming services in case of emergencies and can always be taken everywhere. Behind the frontend, a rule-based clinical decision supports medical staff and the mobile nursing environment. Last but not least, MIMS has been extended to most medical domains and has been integrated with other HIS.

\subsection{RFID applications in hospitals: A case study}

In their conference paper, Wang et al. \cite{casestudy} describe a case study of implementing a RFID system in a Taiwan hospital in the year 2003. The project that was studied was named \ac{LBMS} and was performed in the \ac{TMUH}. In the following section, the development strategy, device management as well as the value generation which are important for developing RFID applications in healthcare organizations, will be discussed. 
Referring to a widely spread disease, called \ac{SARS} in 2003, the authors Wang et al. discuss the effectiveness of applying RFID in hospitals to prevent further infections (e.g. of patients or medical staff). They mention several challenges of implementing RFID systems in hospitals, for instance user/physician resistance, investment problems as well as technical, clinical, organizational and professional resistance. Nevertheless, some hospitals initiated (with subsidies from the Taiwanese government) preliminary RFID projects as early as October 2003 and achieved significant results. 

To give a basic introduction into the existing IT infrastructure in the TMUH, the following paragraph will mention the existing systems of the hospital. TMUH has an integrated HIS that complies with several healthcare standards, such as \ac{HL7}, \ac{DICOM} \cite[p.3 ff.]{casestudy}. Furthermore, the system consists of a LIS, RIS and according to Wang et al. most of the patient's medical records are digitalized. When it comes to the development and the reasons for using LBMS, the authors wanted to build a system that could detect and track potential SARS cases. Medical knowledge and practice should form the basis and core for developing the system. The RFID technology was considered as a tool to support medical practice. In the end, the system should reflect medical assumptions. 
Wang et al. describe a basic workflow with four steps of the LBMS: Initially, all data should be stored in a positioning database which is connected to the existing vital information databases (of the HIS). In the second step, the system automatically retrieves patient medical records from the HIS and runs an inference engine (called 'Rulebase').  'Rulebase' judges whether there was an infectious event or not. If there was a infectious event, the system detects this in a third step. As a consequence (step four) of the detected event, a message is sent immediately to relevant personnel via alarm (email and sms). 
The LBMS can be extended and used in other contexts, like e.g. for precious equipment tracing, in-patient medicine auditing, new-born baby and mother identification or legitimate drug control.
Wang et al. were supported by the Taiwanese government which approved their plan and granted money. As the LBMS should be released as a hospital-wide system, the development required expertise and knowledge from different domains, including medicine, RFID technology, IT systems development, telecommunications and systems integration. Actually, three parties were involved: TMUH, Lion Information Inc. and an advisory group \cite[p.4]{casestudy} which consisted of professors emerging technology and making academic contributions (algorithms).
Since the hospital decided that the system should have active real-time position-tracking, temperature taking and monitoring abilities for tagged patients, the developing team chose 916,5 MHz UHF active tags (see Chapter 2, RFID tags \pageref{tag}) to reduce the risk of staff infections.

Reaching an adequate system integration without loss of performance, functionality and security was a big challenge. With the use of a field generator, a small tag wake-up device that communicates directly with the reader, the real-time communication should be realized \cite[p.4]{casestudy}. The generator periodically turns on and calls tags for a specific time. There exist three different types of generators: Normal, floor and area generators.
Furthermore, Wang et al. bring up the challenge of the entire device management \cite[p.5]{casestudy} with the purpose of collecting and transmitting reads that are as complete and clean as possible. Realizing a complete device management was limited by compartments, rooms, walls and doors because of their buildling layouts and materials which interfere with radiowaves. Besides, the balance between accuracy requirements and investment costs has to be maintained. Moreover, unauthorized removal of tags has to be managed carefully, since there might be some patients who try to take off their RFID wristband. In this case, an additional alarm has to be designed. All in all, the design and deployment of RFID devices depend on the environment and the context in which they are used. 
Not only the device management was challenging but also the data management as Wang et al. mention. The authors describe two general problems of data management in their RFID system. On the one hand, there occur intermittent and unreliable reads. These might be compensated by developing algorithms to process missing and incomplete reads. On the other hand, there will be generated high-volume data in a very short time. To prohibit this, the data should be filtered by algorithms and only the necessary data should be transmitted. For instance, if a tagged patient exceeded the present degree of 0.5\celsius, his data would be transmitted. To come to a conclusion, data management is tied to medical knowledge and practices which can substantially reduce the volume of data to be handled. As a result, meaningful information for decision making will be generated.
 
Besides the LBMS project \cite[p.2 ff.]{casestudy}, Wang et al. depict some existing RFID applications. To give an example of a succesful use of RFID, the U.S. Department of Defence has been using the technology for years. 
To give an overview of the usual hospital applications until 2006, Wang et al. depict applications for tracking and managing equipment such as wheelchairs, portable heart monitors. Moreover, trials on tagging patients, staff and equipment in rooms were conducted in several hospitals. Besides, the Washington Hospital Center (Washington D.C.) deployed a RFID system to track the status and the exact location of patients, staff as well as the essential equipment.
During the realization of the mentioned projects, the solutions depended on building an RFID infrastructure together with the middleware and the impedance-matching of the RFID system and the current systems (e.g. \ac{ERP} systems). Actually, to get along with the mentioned solutions, a strong team work (involving people from IT and business departments) and project management should be included. 
Since RFID allows wireless storage and automatic retrieval of data, there exists an 'ecosystem' of companies trying to develop a platform to support RFID development and applications.  Besides the variety of existing systems in hospitals, Wang et al. mention three mayor technical challenges accomplishing a RFID system. First, the non-line-of-sight reading might be a challenge since there exist various types of tags and the frequencies influence the range of signal. Second, handling of serial numbers might be a challenge which could be coped with setting a primary key to each tag which synchronizes with an existing database (see Chapter 3, 'Used platforms and technologies' \pageref{platforms}. The third challenge is to deal with the real-time data and to synchronize it seasonably. To deal with the third challenge, the use of NoSQL databases makes sense and will be discussed in Chapter 3 \pageref{nosql}.

Finally, Wang et al. evaluate RFID as an infrastructure technology which allows companies to capture data about objects and individuals moving in the real world \cite[p.7]{casestudy}. In addition to that, the authors claim that organizations should think carefully how to change business processes to reap the benefits of RFID. By naming benefits of RFID, Wang et al. refer to the improved efficiency, patient safety and reduced medical errors which can be very extensive and expensive nowadays \cite{casestudy}.

\section{Aim and Scope}

Ajami and Rajabzadeh \cite{ncbi} mention three important purposes of RFID technology. The first purpose of using RFID is to improve the tracking of objects. It is mainly used to follow products through a specific supply chain or to follow medical devices an drugs in the clinical workflow. There is also the possibility to track a product to a particular patient or to identify clinicians who administered medication to patients.
The second purpose for which RFID technology is appropriate is the inventory management (see section \pageref{inventory}). Inventory Management is significant for managing an organization, like a hospital. There are many complex processes where information about the location, time and the amount of material is necessary (e.g. towels, duvet covers).
The third and last purpose of RFID technology, mentioned by Ajami and Rajabzadeh \cite{ncbi}, is validation. Using RFID to identify and validate data is an effective method for ensuring the quality of a hospital or healthcare setting. It ensures that the patient being treated is the right patient.

\subsection{RFID and the IoT}

There exist many applications, which should help us living smarter, not caring about the ordinary things, like for example turning off the washing machine or closing the windows before stepping out. These smart houses form a part of the term \ac{IoT}. Often, the smart solutions are based on RFID technology to identify the exact window or the item that has to be controlled from outside. 
In their book 'RFID Technologies for the Internet of Things', Chen et al. \cite[p.2 f.]{chen} depict smart applications and a specific problem which they call 'tag search problem'. The problem usually appears on large-scale RFID systems and describes the complexity of identifying the wanted tags which exist in the current system. To solve this identification problem, Chen et al. describe the method 'Filtering vectors' which will be explained in the following. 
Firstly, a compact one-dimension bit array is constructed from the tag IDs which are used for filtering the unwanted tags. After that, a novel iterative tag search protocol is run. This protocol progressively improves the accuracy of search results and reduces the time by using information which were detected from previous iterations.
As a second problem of IoT applications, Chen et al. mention the conflict with people's privacy \cite[p.3 f.]{chen}. Since every tag transmits its ID to the nearest reader, the transmission can be exploited by attackers. To prevent eavesdropping, the authors describe a anonymous RFID authentication mechanism which designs anonymous authentication protocols. The protocol is based on cryptographic hash functions which require considerable hardware to randomize the authentication data in order to make the tags untrackable. At this point, one should keep in mind that the provided solution requires valuable hardware and is not suited for low-cost tags which augments the production costs. Thus, manufacturers have to face the challenge of designing anonymous authentication protocols for low-cost tags give limited hardware resources. 
To face the problem of limited hardware resources, Chen et al. suggest an 'asymmetric design principle' \cite[p.4]{chen} which means pushing most of the system's complexity to the reader and leaving the tags as simple as possible. 
Besides the anonymous RFID authentication, tags can be identified by their network \cite[p.4 f.]{chen}. To give an example, in large warehouses there exist a great number of readers and antennas which must be deployed to provide full coverage. To accomplish the full coverage, networked  tags which relay transmissions towards the otherwise-inaccessible reader can be used. As a characteristic of networked RFID tags, they are powered by batteries and rechargeable energy sources (harvest solar, piezoelectric, thermal energy from surrounding environment).
Generally, there can be distinguished two types of ID collection protocols: On the one hand, there is the contention-based ID collection protocol which creates too much overhead in multihop networked tag systems. This leads to an increased collision in the network towards the reader and causes excessive energy costs. On the other hand, Chen et al. mention a serialized ID collection protocol. This solution is based on serial numbers that balance the load and reduce worst-case energy costs. 
As a conclusion, one can say, that imbalanced load in network leads to worst-case energy costs which should be avoided.

\subsubsection{Compact Approximator based Tag Searching protocol}

To avoid the above mentioned energy costs, resulting from inefficient protocols, Chen et al. describe several tag searching protocols \cite[p.13 ff.]{chen} which will be discussed in the following. 
To begin with, one should keep in mind the method 'Filtering vectors' mentioned at the very beginning of this paragraph in which the tag ID was converted into a one-dimension bit array. This first step can be compared with the first step of \ac{CATS}, a two-phased protocol to address the tag identification and it polling problem. The idea of CATS is to encode the tag IDs into a 'Bloom' filter \footnote{A Bloom filter is a compact data structure that encodes membership for a set of items $S=\{ e_{1},e_{2},e_{3},...,e_{n}\}$. To represent S, a bit array of length l is needed. At the beginning, all bits are initialized to zeros. To encode each element $e \in S$, k hash functions are used to map the element randomly to k bits in a bit array, so that the zeros turn into ones.}  \cite[p.15]{chen} and to transmit the Bloom filter instead of the ID. 
Consequently, in the first phase of CATS, the RFID reader encodes all IDs of the wanted tags into a Bloom filter. After encoding, the reader broadcasts the filter together with some parameters to tags in the coverage area. Each tag receives its Bloom filter and tests whether it belongs to  set X. Unwanted tags will be kept silent for the remaining time. Furthermore, a second set Y defines the coverage area of the RFID system. After filtration, the number of candidate tags in Y is reduced.
The second phase of CATS deals with the remaining candidate tags from phase 1. These tags report their particular Bloom filter during several time slots. Each candidate tag transmits in k slots and is mapped to a certain set. During the transmission, the reader is listening to the channel and builds a second Bloom filter based on the status of time slots: '1' stands for busy slot which means that at least one tag is transmitting whereas '0' stands for idle slot during which no tag is transmitting. 
These two phases build the main activities of the CATS protocol and seem to be realized very easily. Nevertheless, Chen et al. introduce some raising problems by using CATS. One problem is optimizing the Bloom filter sizes since CATS approximates two Bloom filters together as the first, so that $|X \cap Y|=|X|$. A second problem is that CATS assumes that the first Bloom filter is always smaller than the second one: $|X|<|Y|$. But in reality, the number of wanted tags may be far greater than the number in the coverage area of the RFID system. 

\subsubsection{Iterative Tag Search Protocol}

To avoid the errors caused by using CATS, Chen et al. describe another effective tag search protocol which is called \ac{ITSP} \cite[p.22-28]{chen}. Assuming that there is a wireless channel available between the RFID reader and the tag, ITSP interferes from nearby equipment (e.g. motors, conveyers, robots, WLANs, cordless phones). Furthermore, ITSP divides the bidirectional filtration of the tag search process into multiple rounds. Before each round i, a set of candidate tags in X is denoted as $X_{i}(\subseteq X)$ which represents the search result after $(i-1)$ round. Thus, the final search result is a set of remaining candidate tags in X after all rounds are completed. So, ITSP can be seen as a general iterative approach allowing multiple filtering vectors to be sent consecutively. Each round contains two phases. During the first phase, the RFID reader constructs $m_{i}$ filtering vectors for $X_{i}$ using $m_{i}$ hash functions \cite[p.22]{chen}. In a second step, the reader broadcasts the filtering vectors one by one and each tag receives its own filtering vector. By checkiing its ID with the filtering vector, each tag uses the same hash function as the reader. As a result, each tag can get a '1' which means that is it a candidate tag of $Y_{i+1}$  or it receives a '0' which excludes the tag and drops it out of the search process.

Afterwards, during the second round, the reader broadcasts the frame size $L_{Y_{i+1}}$ to the tags (which are all candidate tags) and each tag does the same as in round one. After receiving its filtering vector, each tag randomly maps its ID to a slot in the time frame using a hash function and transmits a response to the reader $(0 or 1)$. After receiving the response from each tag, the reader constructs a new filtering vector which is used to filter the non-candidate tags from $X_{i}$.

After the two phases the reader updates the current stage which contains a set of remaining candidate tags. The number of tags shrinks from $X_{i}$ to $X_{i+1}$ during this step. 

\subsubsection{Lightweight Anonymous RFID Authentication}

There exist many different authentication mechanisms in RFID applications. To give an example of one current possibility to authenticate RFID tags, Chen et al. depict the 'Lightweight Anonymous RFID Authentication' \cite[p.39 ff.]{chen}. To start with, a fundamental system model is given (see figure~\ref{fig:lightweight}). In addition to the system model, it should be noticed that each tag is pre-installed with some keys for authentication. Furthermore, all readers are deployed at chosen locations and connected to backend servers which are connected to a central server. On the central server, every tag's keys are stored.

\begin{figure}
\centering
\includegraphics[width=\textwidth]{lightweight} The system model of Lightweight Anonymous RFID Authentication
\caption{\label{fig:lightweight}\cite[p.40]{chen}} 
\end{figure}

Concerning the communication between readers and tags, Chen et al. describe a 'Request-Response mode Communication' \cite[p.40 ff.]{chen}. Firstly, the reader by initiating the communication it sends a request to the tag. Secondly, after receiving the request, the tag makes an appropriate transmission as response. There can be distinguished two types of transmissions: Invariant and variant transmissions. The first type (invariant transmissions) can contain content that is 'invariant' between the tag and the reader. In contrast to that, variant transmissions can contain content that may vary for different tags or the same tag at different times, e.g. when it contains exchanged data for anonymous authentication.  


\subsubsection{Identifying state-free networked tags}

Before explaining the mechanism of identifying state-free networked tags, the two terms 'state-free' and 'stateful' networked tags should be made clear. If a networked tag is called stateful, it maintains its network state which includes information about its neighbors in the network, routing tables as well as update information to keep it up-to-date. In contrast to that, state-free tags serve the purpose of energy conservation and do not maintain any network state prior to their operations which makes them different from traditional networks. 
Given the two definitions, there comes up the challenge of identifying state-free networked tags. Chen et al. refer to that challenge and explain a method for identifying networked tags \cite[p.67 ff.]{chen} which will be discussed in the following. 
First of all, all tags in a network are connected to each other through a peer communication (to the nearby tags). Especially the emerging number of networked tags represent a significant enhancement to today's RFID technology. The problem of readers which cannot cover all tags due to cost or physical limitations is solved by using networked tags because the ones near to each other can directly communicate. Moreover, the possibility of peer communication enables a multihop network to be formed among the tags. The transmission range of inter-tag communications is usually short and amounts to about 1-10 m whereas the transmission range of a reader is much larger. 
Nevertheless, the peer communication realizes a direct two-way communication between the current node and the neighboring node. Concerning the energy input, the energy can be powered through the reader's radio waves but the internal energy should be carried sufficiently for long-term operations (must be made energy-efficient).
After establishing a peer-to-peer communication, the reader collects the IDs from all networked tags that are in its read range. Using multiple hops and intermediate tags relaying the IDs of those tags which are not in the immediate coverage area of the reader, state-free tags can be identified. 































%uhf, reaching signal -->see chapter \chapterref{Kap2}