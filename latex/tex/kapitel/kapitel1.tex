\chapter{Use of RFID Technology}
\label{Kap1}

The following chapter will discuss the reasons why choosing the technology and the scope of the RFID technology. It will start by explaining fundamental information and functionalities of RFID readers, tags and further equipment to build a RFID application. After that, some 'State of the Art' applications and use cases will be shown. In the end, there will be mentioned some large companies which provide medical RFID solutions.

\section{Motivation}

Concerning the organization and management of medical devices or patients in a hospital, there exist many problems. In the following, some examples of these problems as described by Ajami and Rajabzadeh \cite{ncbi} will be given.

\subsection{Decision Making}

First of all, when it comes to decision making, e.g. about the correct treatment of a severe illness, many physicians are stumped for an answer or their opinions are divided. To enable a rapid diagnosis and to improve the patient's health status, 'smart healthcare' \cite{henrici} would help a lot. For instance, RFID tags which are equipped with sensors to ensure the effectiveness of medicine accelerate the treatment process a lot. Furthermore, patients or hospital beds equipped with RFID tags make it easier to identify and manage the amount of patients as well as the workflow.

\subsection{Internal Communication}

Secondly, poor communication between nurses and physicians deteriorates medical supply. For instance, if a nurse notices that a patient needs a more tranquilizer because he became very nervous, she has to tell the doctor to dose the patient with the correct amount of tranquilizer. But often, a physician is occupying another patient. So, the communication is one problem but the other problem is the staff shortage. Thus, inadequate patient monitoring is emerges. And sometimes, there is the risk of misidentification of patients. To explain the last point, there is an easy example: At the urology department are two elder patients, Paul Schmitt and Jochen Schmitt. They are not brothers or related to each other and suffer from different types of illness. Paul suffers from kidney insufficiency whereas Jochen suffers from prostatic lithiasis. The first one needs a dialysis every day whereas the second one needs a radiosurgery. Because both patients are unable to walk themselves, nurses and clinical staff have to bring them to the particular treatment room. The problem should be easy to understand, both patients have the same surname but need completely different treatments. If the treatments would be commuted, their health status would deteriorate and they might die because of the misidentification.

\subsection{Investment possibilities}

Another important point for hospitals is the budget and their possibilities to investigate in new technologies which makes the enrollment of a new RFID system more challenging. Furthermore, clinical staff and phycisians have to be introduced into the new technologies. Not only the human factor plays a significant role but also the existing systems, such as the \ac{HIS}, \ac{RIS} or \ac{LIS}. If a new identifying system or software should be integrated into a hospital or healthcare institution, it has to be deployed suitably to the existing system architecture. To achieve the last point, Ajami and Rajabzadeh \cite{ncbi} recommend starting with small RFID projects and mentions countermeasures to increase the acceptance of such applications by healthcare institutions. To give an example, the regulations to protect patient's privacy should be mature to achieve more institutional support. Besides, there should exist more customized RFID systems which accomplish the individual tasks of their users.  

\subsection{Medication Administration System}

To explain the positive impact of using RFID systems, in the following, a few applications will be described briefly (see also \cite{ncbi}). Firstly, Ajami and Rajabzadeh describe a Medication Administration System which automatically verifies medication and generates the corresponding medication medicine. There exists multiple intents of developing a Administration System, such as preventing human errors (like for example mislabeling of tissue specimens in gastrointestinal and colorectal surgery endoscopy units). The second most common error which occured were that patients have been labelled incorrectly. To avoid these errors, an initiative of developing an application of RFID technology to specimen bottles was started. The aim of this initiative was to create a paperless pathology requisition system and the correct confirmation by both the endoscopy nursing staff as well as the endoscopist for each specimen bottle. As a result of deploying the application, specimen-labeling errors were significantly reduced.

\subsection{Wisely Aware RFID Dosage}

Another RFID system, called \ac{WARD} system should prevent the risk of medication errors triggered by medical staff. It is based on integrated barcode and RFID tags which should demonstrate effective and safe patient care environment. 
To give another example of RFID applications, the following paragraph will describe the \ac{MIMS} which includes a mobile nursing care system using RFID technology. There are many implemented functionalities in the MIMS, such as the tracking of patient's vital signs across various locations and in different medical facilities. The vital sign monitoring enables medical staff to watch critical ill patients carefully and permanently and reduces the risk of serious harm resulting from slow provision \cite{ncbi}. The mobile application MIMS offers alarming services in case of emergencies and can always be taken everywhere. Behind the frontend, a rule-based clinical decision supports medical staff and the mobile nursing environment. Last but not least, MIMS has been extended to most medical domains and has been integrated with other HIS.

\subsection{RFID applications in hospitals: A case study}

In their conference paper, Wang et al. \cite{casestudy} describe a case study of implementing a RFID system in a Taiwan hospital in the year 2003. The project that was studied was named \ac{LBMS} and was performed in the \ac{TMUH}. In the following section, the development strategy, device management as well as the value generation which are important for developing RFID applications in healthcare organizations, will be discussed. 
Referring to a widely spread disease, called \ac{SARS} in 2003, the authors Wang et al. discuss the effectiveness of applying RFID in hospitals to prevent further infections (e.g. of patients or medical staff). They mention several challenges of implementing RFID systems in hospitals, for instance user/physician resistance, investment problems as well as technical, clinical, organizational and professional resistance. Nevertheless, some hospitals initiated (with subsidies from the Taiwanese government) preliminary RFID projects as early as October 2003 and achieved significant results. 

To give a basic introduction into the existing IT infrastructure in the TMUH, the following paragraph will show existing systems. 
 
in the LBMS project -->\cite[p.3 ff.]{casestudy}

Besides the LBMS project \cite[p.2 ff.]{casestudy}, Wang et al. depict some existing RFID applications. To give an example of a succesful use of RFID, the U.S. Department of Defence has been using the technology for years. 
To give an overview of the usual hospital applications until 2006, Wang et al. depict applications for tracking and managing equipment such as wheelchairs, portable heart monitors. Moreover, trials on tagging patients, staff and equipment in rooms were conducted in several hospitals. Besides, the Washington Hospital Center (Washington D.C.) deployed a RFID system to track the status and the exact location of patients, staff as well as the essential equipment.
During the realization of the mentioned projects, the solutions depended on building an RFID infrastructure together with the middleware and the impedance-matching of the RFID system and the current systems (e.g. \ac{ERP} systems). Actually, to get along with the mentioned solutions, a strong team work (involving people from IT and business departments) and project management should be included. 
Since RFID allows wireless storage and automatic retrieval of data, there exists an 'ecosystem' of companies trying to develop a platform to support RFID development and applications.  Besides the variety of existing systems in hospitals, Wang et al. mention three mayor technical challenges accomplishing a RFID system. First, the non-line-of-sight reading might be a challenge since there exist various types of tags and the frequencies influence the range of signal. Second, handling of serial numbers might be a challenge which could be coped with setting a primary key to each tag which synchronizes with an existing database (see Chapter 3, 'Used platforms and technologies' \pageref{platforms}. The third challenge is to deal with the real-time data and to synchronize it seasonably. To deal with the third challenge, the use of NoSQL databases makes sense and will be discussed in Chapter 3 \pageref{nosql}.

\section{Aim and Scope}

Ajami and Rajabzadeh \cite{ncbi} mention three important purposes of RFID technology. The first purpose of using RFID is to improve the tracking of objects. It is mainly used to follow products through a specific supply chain or to follow medical devices an drugs in the clinical workflow. There is also the possibility to track a product to a particular patient or to identify clinicians who administered medication to patients.
The second purpose for which RFID technology is appropriate is the inventory management (see section \pageref{inventory}). Inventory Management is significant for managing an organization, like a hospital. There are many complex processes where information about the location, time and the amount of material is necessary (e.g. towels, duvet covers).
The third and last purpose of RFID technology, mentioned by Ajami and Rajabzadeh \cite{ncbi}, is validation. Using RFID to identify and validate data is an effective method for ensuring the quality of a hospital or healthcare setting. It ensures that the patient being treated is the right patient.

uhf, reaching signal -->see chapter \chapterref{Kap2}







