% Die längste Abkürzung kann in die eckigen Klammern
% bei \begin{acronym} geschrieben, um einen häßlichen
% Umbruch zu verhindern
\begin{acronym}[WORM]
\acro{RFID}{Radio Frequency Identification}
\acro{NFC}{Near Field Communication}
\acro{IoT}{Internet of Things}
\acro{CT}{Computer Tomograph}
\acro{RTI}{Reusable Transport Items}
\acro{IT}{Information Technology}
\acro{UHF}{Ultra High Frequency}
\acro{RAIN}{RAdio Frequency IdentificatioN}
\acro{HIS}{Hospital Information System}
\acro{RIS}{Radiology Information System}
\acro{LIS}{Laboratory Information System}
\acro{EPC}{Electronic Product Code}
\acro{WORM}{Write Once Read Many}
\acro{WARD}{Wisely Aware RFID Dosage}
\acro{MIMS}{Mobile Intelligent Medical System}
\acro{LF}{Low Frequency}
\acro{HF}{High Frequency}
\acro{UHF}{Ultra High Frequency}
\acro{SSL}{Secure Sockets Layer}
\acro{TLS}{Transport Layer Security}
\acro{FDA}{Federal Drug Administration}
\acro{SMLE}{Single Logical Message Exchange}
\acro{SARS}{Severe Acute Respiratory Syndrome}
\acro{LBMS}{Location-based Medical Service}
\acro{TMUH}{Taipei Medical University Hospital}
\acro{ERP}{Enterprise Resource Planning}
\acro{HL7}{Health Level 7}
\acro{DICOM}{Digital Imaging and Communications in Medicine}
\acro{IoT}{Internet of Things}
\acro{CATS}{Compact Approximator based Tag Searching protocol}
\end{acronym}
