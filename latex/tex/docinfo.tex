% -------------------------------------------------------
% Daten für die Arbeit
% Wenn hier alles korrekt eingetragen wurde, wird das Titelblatt
% automatisch generiert. D.h. die Datei titelblatt.tex muss nicht mehr
% angepasst werden.

\newcommand{\hsmasprache}{en} % de oder en für Deutsch oder Englisch
% Für korrekt sortierte Literatureinträge, noch preambel.tex anpassen
% und zwar bei \usepackage[main=ngerman, english]{babel},
% \usepackage[pagebackref=false,german]{hyperref}
% und \usepackage[autostyle=true,german=quotes]{csquotes}

% Titel der Arbeit auf Deutsch
\newcommand{\hsmatitelde}{Entwicklung von RFID Anwendungen zur Verwaltung von Beständen und Medikamenten in Krankenhäusern}

% Titel der Arbeit auf Englisch
\newcommand{\hsmatitelen}{Development of RFID applications for management and tracking assets and drugs in hospitals}

% Weitere Informationen zur Arbeit
\newcommand{\hsmaort}{Mannheim}    % Ort
\newcommand{\hsmaautorvname}{Jacqueline} % Vorname(n)
\newcommand{\hsmaautornname}{Franßen} % Nachname(n)
\newcommand{\hsmadatum}{31.08.2018} % Datum der Abgabe
\newcommand{\hsmajahr}{2018} % Jahr der Abgabe
\newcommand{\hsmafirma}{Escuela Politécnica de Ingeniería de Gijón} % Firma bei der die Arbeit durchgeführt wurde
\newcommand{\hsmabetreuer}{Prof. Dr. Miriam Föller-Nord, Hochschule Mannheim} % Betreuer an der Hochschule
\newcommand{\hsmazweitkorrektor}{Prof. Dr. Thomas Smits, Hochschule Mannheim} % Betreuer im Unternehmen oder Zweitkorrektor
\newcommand{\hsmafakultaet}{I} % I für Informatik
\newcommand{\hsmastudiengang}{IMB} % IB IMB UIB IM MTB

% Zustimmung zur Veröffentlichung
\setboolean{hsmapublizieren}{true}   % Einer Veröffentlichung wird zugestimmt
\setboolean{hsmasperrvermerk}{false} % Die Arbeit hat keinen Sperrvermerk

% -------------------------------------------------------
% Abstract

% Kurze (maximal halbseitige) Beschreibung, worum es in der Arbeit geht auf Deutsch
\newcommand{\hsmaabstractde}{Die folgende Bachelorthesis beschreibt die Entwicklung einer mobilen, hybriden Anwendung für Android- und iOS-Geräte. Das entwickelte System (Hardware und Software) zielt die Verfolgung und Verwaltung von medizinischen Geräten sowie Arzneimitteln in Krankenhäusern an.
Im Vergleich zu bisherigen Arbeiten behandelt die vorliegende Arbeit unter anderem die ''Real-time''-Synchronisierung, realisiert durch Socket.IO, zwischen Smartphones und dem Server im Krankenhaus. Dadurch wird ein möglichst synchroner Lösungsansatz verfolgt, um zeitnah auf etwaige Mängelbestände der Medikamente reagieren zu können.

Das vorliegende Projekt beabsichtigt die Entwicklung eines RFID basierten Systems für Verwaltungs- und Verfolgungsanwendungen. Insbesondere um die Kontrolle der Medikamenten, welche regelmäßig Patienten verabreicht werden, sowie medizinischer Geräte zu verbessern, soll das System in Krankenhäusern eingesetzt werden. Ein weiterer Benefit des vorgestellten System ist die Reduzierung 
von vergessener oder falscher Medikamentenverabreichung.
Das behandelte System besteht aus RFID Antennen, welche die Reichweite des Systems (in welcher RFID Tags erkannt werden) festlegen.
Diese Antennen und der RFID Leser sind verbunden mit einer Datenbank, welche Informationen zu den Gegenständen und Medikamenten enthält. Das Projekt deckt insbesondere Themen ab, die von großem Interesse für medizinische und \ac{ITC} Firmen sind, wie beispielsweise Internet of Things und e-Health.}

% Kurze (maximal halbseitige) Beschreibung, worum es in der Arbeit geht auf Englisch

\newcommand{\hsmaabstracten}{The following Bachelor Thesis describes the on the development of a mobile hybrid application which can be run both on Android and iOS devices as well as the physical layer hardware based on RFID technology. The developed system (both hardware and software) is devoted to  be used for tracking and management of medical assets and drugs in hospitals. Compared to other existing systems, the given thesis includes the development of 'real-time' synchronization between client and server, realized by using Socket.IO. Thus, a very synchronous solution approach is pursued in order to react contemporarily to missing drugs.

This project aims to develop a RFID-based system for management and tracking applications. In particular, the proposed system is devoted to be deployed at hospitals, to ensure a better control of the medicines and medical tools that are periodically administered to the patients, reducing the probability of missed or wrong medicine administration. The proposed system is formed by a set of RFID antennas that will conform the coverage area where items tagged with RFID tags will be read. These antennas and the RFID reader will be connected to a database that will contain the information of the items to be monitored. The project covers topics of great interest for medical and \ac{ITC} companies, such as IoT and e-Health.}
