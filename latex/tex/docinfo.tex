% -------------------------------------------------------
% Daten für die Arbeit
% Wenn hier alles korrekt eingetragen wurde, wird das Titelblatt
% automatisch generiert. D.h. die Datei titelblatt.tex muss nicht mehr
% angepasst werden.

\newcommand{\hsmasprache}{en} % de oder en für Deutsch oder Englisch
% Für korrekt sortierte Literatureinträge, noch preambel.tex anpassen
% und zwar bei \usepackage[main=ngerman, english]{babel},
% \usepackage[pagebackref=false,german]{hyperref}
% und \usepackage[autostyle=true,german=quotes]{csquotes}

% Titel der Arbeit auf Deutsch
\newcommand{\hsmatitelde}{Entwicklung einer mobilen RFID-Anwendung zum Arznei-Management in Krankenhäusern und Apotheken}

% Titel der Arbeit auf Englisch
\newcommand{\hsmatitelen}{Development of a mobile application with the technology of RFID for managing drugs in hospitals and pharmacies}

% Weitere Informationen zur Arbeit
\newcommand{\hsmaort}{Mannheim}    % Ort
\newcommand{\hsmaautorvname}{Jacqueline} % Vorname(n)
\newcommand{\hsmaautornname}{Franßen} % Nachname(n)
\newcommand{\hsmadatum}{31.08.2018} % Datum der Abgabe
\newcommand{\hsmajahr}{2018} % Jahr der Abgabe
\newcommand{\hsmafirma}{Escuela Politécnica de Ingeniería de Gijón} % Firma bei der die Arbeit durchgeführt wurde
\newcommand{\hsmabetreuer}{Prof. Dr. Miriam Föller-Nord, Hochschule Mannheim} % Betreuer an der Hochschule
\newcommand{\hsmazweitkorrektor}{Prof. Dr. Thomas Smits, Hochschule Mannheim} % Betreuer im Unternehmen oder Zweitkorrektor
\newcommand{\hsmafakultaet}{I} % I für Informatik
\newcommand{\hsmastudiengang}{IMB} % IB IMB UIB IM MTB

% Zustimmung zur Veröffentlichung
\setboolean{hsmapublizieren}{true}   % Einer Veröffentlichung wird zugestimmt
\setboolean{hsmasperrvermerk}{false} % Die Arbeit hat keinen Sperrvermerk

% -------------------------------------------------------
% Abstract

% Kurze (maximal halbseitige) Beschreibung, worum es in der Arbeit geht auf Deutsch
\newcommand{\hsmaabstractde}{In der folgenden Arbeit wird die Entwicklung einer mobilen, hybriden Anwendung für Android- und iOS-Smartphones beschrieben. Die Anwendung wurde für den Einsatz in Krankenhäusern und Apotheken entwickelt. Das User-Szenario beinhaltet die Erfassung, Verfolgung und Verwaltung von medizinischen Arzneimitteln und Medikamenten, welche mittels der RFID-Technologie realisiert wurde.}

% Kurze (maximal halbseitige) Beschreibung, worum es in der Arbeit geht auf Englisch

\newcommand{\hsmaabstracten}{The following thesis is focussed on the development of an mobile hybride application which can be run on Android as well as on iOS devices. The application is specialized in the use in hospitals and pharmacies. The scope of the application contains the registration, tracking as well as the management of pharmaceuticals and drugs, realized by the technology of RFID.}
